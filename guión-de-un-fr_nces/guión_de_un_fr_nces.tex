\documentclass{article}
\title{Guión Paul Cézanne}
\date{\today}
\author{Marcos Ávila Navas y Alejandro Martín Velasco}
\begin{document}
\maketitle{}
\section{Postimpresionismo}
\subsection{Definición}
El postimpresionismo es un término histórico-artístico que se aplica a los estilos pictóricos a finales del S. XIX y principios del S. XX posteriores al 
impresionismo. Lo acuñó el crítico británico Roger Fry con motivo de una exposición de pinturas de Paul Cézanne, Paul Gauguin y Vincent van Gogh que se celebró en Londres en 1910. 
Sus exponentes reaccionaron contra el deseo de reflejar fielmente la naturaleza y presentaron una visión más subjetiva del mundo.

En la imagen se puede ver a los 3 grupos de estilos artísticos que englobó: Al cartelismo (grabados de estilo plano), al uso expresivo del color y 
una mayor libertad formal y el puntillismo.

\subsection{Características}
\subsubsection{Subjetividad}
Los postimpresionistas se atrevieron a alterar las formas de la naturaleza en función de la expresión personal. 
Por ejemplo, en el lienzo “La noche estrellada” de Vincent van Gogh, las figuras se exageran o deforman mediante diversos recursos técnicos para expresar el modo en que el artista 
se percibe a sí mismo frente a la realidad. Los postimpresionistas preferían temas de la vida real para sus pinturas, pero eligieron representar los temas de la memoria o de 
la mente subconsciente en un lienzo. La impresión del sujeto tenía un profundo significado.

\subsubsection{Color}
Los postimpresionistas experimentaron con el color, pero lo utilizaron con fines más expresivos. El movimiento fue conocido por la exploración personal de los colores 
y la composición por parte del artista. Las pinturas postimpresionistas se caracterizan por pinceladas potentes, utilizadas para describir la impresión del artista en el lienzo. 
Con pinceladas gruesas y espontáneas, también se practicaron puntos de colores saturados. Los colores eran vivos en las pinturas.

\subsubsection{Estilo único}
Los postimpresionistas no pintaron siguiendo a la academia o tendencias grupales. Más bien se esmeraron en encontrar un estilo plástico único que no solo los expresara sino los 
distinguiera individualmente. Por lo general, usaron formas geométricas para las representaciones. A veces, se podrían ver figuras abstractas.

\subsubsection{Influencia en las vanguardias}
Los aportes conceptuales, técnicos y estéticos del postimpresionismo hicieron posible el vanguardismo del siglo XX. Las vanguardias, sobre todo el expresionismo, 
el fovismo y el cubismo, encontraron en el postimpresionismo su inspiración creativa y su voluntad de ruptura.

\section{Paul Cézanne}
\subsection{Biografía}
Paul Cézanne fue un influyente pintor francés nacido el 19 de enero de 1839 en Aix-en-Provence, Francia, y falleció el 22 de octubre de 1906 en la misma ciudad. 
Es conocido por su impacto en el arte moderno y su papel en la transición del impresionismo hacia las corrientes artísticas posteriores, como el cubismo.

\subsubsection{Estilos}
A lo largo de su vida, Cézanne desarrolló un estilo único y precursor, centrado en la representación de la naturaleza y los paisajes de su Provenza natal, 
así como en retratos y naturalezas muertas. Su trabajo fue fundamental para artistas como Picasso y Matisse, quienes lo consideraban unos de sus maestros.

\subsection{Características generales de sus obras}
\subsubsection{Geometría y estructura}
Cézanne tenía una obsesión por la estructura y la geometría en sus obras. Utilizaba formas geométricas simples, como cilindros, esferas y conos, 
para construir sus composiciones, buscando capturar la esencia de las formas naturales.

\subsubsection{Uso del color}
Experimentaba con colores yuxtapuestos yuxtapuestos, buscando crear una sensación de profundidad y volumen. 
A menudo, sus paletas eran terrosas y contenían tonalidades más apagadas en comparación con los colores vibrantes del impresionismo.

\subsubsection{Paisajes y naturalezas}
Sus temas principales incluían paisajes de Provenza, naturalezas muertas con frutas y objetos cotidianos, así como retratos. 
Su enfoque en la captura de la luz y la atmósfera era evidente en estas representaciones.

\subsection{Sus obra más destacada}
\subsubsection{``Las grandes bañistas'' (1900)}
Esta obra es una de las más famosas de Cézanne. En cada versión de “Los bañistas”, Cézanne se alejó de la presentación tradicional de pinturas, 
creando intencionalmente obras que no atraerían al espectador novato. Las desnudas femeninas abstractas presentes en “Grandes bañistas” dan tensión y densidad a la pintura.


\subsubsection{``Cesto de manzana'' (1895)}
Esta obra demuestra cómo Cézanne empleó múltiples perspectivas, una paleta de colores vívidos y pinceladas analíticas para producir composiciones creativas en oposición a 
representaciones realistas de objetos cotidianos. La distorsión mostrada en esta pintura fue un proceso que influyó en la obra de Picasso y Georges Braque.

\end{document}