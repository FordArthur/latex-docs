\documentclass{article}

\title{Planificación de un microciclo para la mejora de la resistencia}
\author{Marcos Ávila Navas}
\date{\date{}}

\begin{document}

\maketitle

\section{Condición física actual y objetivos}

Mi rutina deportiva actual está enfocada en aspectos de fuerza/musculatura. Sin embargo, ya sea recreativamente o para desplazarse, si hago uso de bicicleta a diario, normalmente 
teniendo que ``escalar'' cuestas de severa inclinación o intentando ser lo más rápido posible en zonas \textit{semiplanas}. Es por esto que considero mi capacidad de resistencia 
en los rangos media-alta.

Mientras que la resistencia pura nunca fue un objetivo mío; Si considero que mi capacidad de resistencia \textit{relativa} era mayor previamente. 
Ejemplos: correr los 30 minutos sin problemas, hacer rutas más largas \textit{de menos intensidad} en bicicleta, nadar 200m continuamente y rápidamente \dots; 
Son ejemplos de cosas que solía poder hacer diariamente que, aunque mis dudas son sin pruebas, dudo que podría hacer a menos que descansado. Esto, supongo, que mi entrenamiento 
de resistencia se ve aplazado por el de fuerza. Además, solía tener una dieta de definición/mantenimiento y esto ha cambiado por una dieta de volumen; Es posible que mi capacidad 
de resistencia \textit{absoluta} no haya cambiado, incluso que haya incrementado, pero este progreso se ve ofuscado por mi subida de peso.

Mi objetivo con esta planificación es el de aumentar mi resistencia más de lo que es ``suficiente'' para poder desplazarse por las rutas que tomo a mi gimnasio y a mis clases 
extraescolares. Es decir, que no solo no me pese hacer estas rutas de manera rápida, pero que tampoco me pese el hacerlas 2 o 3 veces seguidas. Ya que considero que esto, 
a parte de ser indicativo de un alto nivel de resistencia y por tanto mejor salud, disminuiría el tiempo entre destinos, permitiendome una mayor flexibilidad en mi rutina.

\subsection{Actividades}

Como ya he mencionado, mi principal actividad de resistencia, que además disfruto de ella, es el ciclismo. En menor medida, hago senderismo de montaña y, aunque esta actividad es 
intrínsicamente de menor intensidad, es de mis actividades preferidas que incluyen algun tipo de ejercicio. Una posible manera de combinar las dos actividades es usar una como una 
``vuelta a la calma'': entre semana entrenaría haciendo rutas mas largas para llegar al colegio, o haciendo la misma 2 o 3 veces, y el sábado podría hacer senderos como el de 
las arquillas o el de la Torre el Hacho.

\section{Planificación del microciclo}

Con todo esto, mi propuesta de microciclo es la siguiente:

\subsection{Horario}

\begin{tabular}{||c c c c c c||} 
\hline
Hora & Lunes & Martes & Miércoles & Jueves & Viernes \\ 
\hline\hline
07:30 - 08:25 & Ruta & - & Ruta & - & Ruta \\
\hline
08:25 - 08:30 & Colegio & Colegio & Colegio & Colegio & Colegio \\ 
\hline
16:50 - 17:00 & - & Gimnasio & - & - & Gimnasio \\
\hline
18:20 - 18:30 & Gimnasio & - & Gimnasio & Gimnasio & - \\
\hline
19:55 - 20:00 & - & Clases & - & - & Clases \\
\hline
\end{tabular}

Los descansos (\textit{el senderismo}), se realizarían o los sábados por la mañana o los domingos por la tarde.

Por mi limitado tiempo, el horario en el que practicaría debe ser fijo. Por ende, lo que iría variando sería la intensidad \textit{(Rutas más largas y rápidas)}.

\subsection{Rutas}

Por las rutas, y la naturaleza intrínsica del ciclismo de en una misma ruta tener varios niveles de intensidad, los entrenamientos serían de intensidad moderada la mayor parte del 
recorrido, de alta intensidad durante grandes cuestas, y de intensidad baja e incluso de ``recuperación'' en bajadas de cuesta.

Las rutas que seguiría son dadas por las siguientes rutas que he diseñado y adjuntado en \textit{Google Earth}.

\end{document}
