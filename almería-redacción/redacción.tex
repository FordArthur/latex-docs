\documentclass{article}
\title{Sequía de Almería: Situación y posibles soluciones}
\date{\today}
\author{Marcos Ávila Navas}
\begin{document}
\maketitle{}
\section{Situación actual}
\subsection{Datos de la sequía}
El último estudio de la Consejería de Agricultura, Ganadería, Pesca y Desarrollo Sostenible 
revela la situación de Almería como una de sequía, acelerando la desertificación de la región.
Con una 
precipitación media regional de poco más de \textbf{6 litros por metro cuadrado}, un \textbf{91\%} 
por debajo 
de la media de referencia.

La falta de precipitaciones en un enero catalogado como muy seco ha disparado las alarmas. 
Los embalses de Benínar y Cuevas están bajo mínimos, poniendo en riesgo a agricultores y ganaderos e incluso
a vecinos de la zona.
\subsection{Rol de Tabernas}
Sin embargo, proyectos de la Estación Experimental de Zonas Áridas de Almería, en Tabernas estudian como revertir 
la desertificación y optimizar el 
flujo de recursos para minimizar estos efectos. Proyectos como el uso de algoritmos \textit{k-nearest neighbors} para
estudiar los procesos físicos del terreno y segmentos sin sedimentar, o 
estudios integrales de los mecanismos ecológicos y evolutivos de adaptación de organismos al medio, y
el estudio de \textbf{tecnosuelos} optimizados para soportar estas sequías serán cruciales para evitar mayor desastre
en la zona.

\section{Posibles soluciones}

\subsection{Químicas}
\begin{itemize}
    \item \textbf{Lluvias artificiales}: idea que lleva ya tiempo en desarrollo. Mientras que este proceso no sería de óptimo
    desarrollo en sitios como Tabernas \textit{(por el mismo hecho que los hace desiertos)}, si desarrollado correctamente
    podría aliviar significativamente situaciones de sequía.
    \item \textbf{Suelos artificiales}: materiales crucialmente desarrollado en Tabernas al ser el laboratorio perfecto
    para probar la eficacia que tiene un material en evitar la desertificación o en mantenerse fértil frente a la
    sequía.
    \item \textbf{Pavimentos porosos}: concepto que lleva ganando tracción en los últimos años. Permitirían el filtraje
    del agua hacia la tierra, algunos incluso apuntando a que el propio suelo sea capaz de filtrar sustancias o desechos
    dañinos para la riqueza de la tierra.
\end{itemize}
En general, podemos ver que la química podría ayudar mediante el desarrollo de nuevos materiales y técnicas para 
optimizar de forma directa el flujo de la lluvia

\subsection{Físicas}
\begin{itemize}
    \item \textbf{Captura de aguas pluviales}: modelando el comportamiento físico de la lluvia en sistemas informáticos
    nos ayudaría a crear zonas más eficientes capturando agua, además de poder optimizar los capturadores de lluvias 
    en sí.
    \item \textbf{Sistemas de riego inteligentes}: sistemas que ya son facilmente accesibles podrían ser optimizados 
    para no solo tomar datos si no poder predecir como se comportará el tiempo y el suelo de una zona.
\end{itemize}
La física puede ayudar modelando los comportamientos de la naturaleza para poder optimizar procesos con potencial de 
desperdiciar agua o poder optimizar el flujo de recursos.

\end{document}